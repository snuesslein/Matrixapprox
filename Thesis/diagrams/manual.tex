\documentclass[inputenc=utf8,fontsize=10pt]{article}
\usepackage[utf8]{inputenc}

\usepackage[scaled]{helvet}
\renewcommand*\familydefault{\sfdefault} %% Only if the base font of the document is to be sans serif
\usepackage[T1]{fontenc}

\usepackage{amsmath}
\usepackage{amsfonts}
\usepackage{amssymb}
\usepackage{graphicx}
\author{Stephan Nüßlein}
\title{Time Varying Systems in Tikz}

\usepackage{pgfplots}
\usepackage{tikz}
\usetikzlibrary{calc}
\usetikzlibrary{matrix}
\usetikzlibrary{ decorations.markings}
\usetikzlibrary {decorations.shapes}


%Standard settings for the stage4s
\tikzset{
	style_stage/.style={
		color=black!80, %line color
		draw,
		fill=white,
		line width=1pt,
	}
}

%special parametres for stages
\tikzset{fontmatrices/.initial=\small}
\tikzset{boxsize/.initial=9mm} %Note: width=height=2*boxsize
%psoition of arrow
\tikzset{posarr/.initial=0.5}

%line style for the signals
\tikzstyle{signal} = [line width=1.5pt]%[very thick]

%line style for drawing connections
\tikzset{signalflow/.style={signal,
		decoration={markings,mark=at position \pgfkeysvalueof{/tikz/posarr} with {\arrow{>}}},postaction={decorate}}}

%set the standard parameters for A,B,C,D
\tikzset{A/.initial=$A_{}$}
\tikzset{B/.initial=$B_{}$}
\tikzset{C/.initial=$C_{}$}
\tikzset{D/.initial=$D_{}$}


%Some internal definitions
\def\rel_i_stage{0.65} %distance of the connections for B and C, is a fraction of boxsize
\def\shadesize{2pt}

%some shorthanfds for convenience
\def\boxsize{\pgfkeysvalueof{/tikz/boxsize}}

\makeatletter
\pgfdeclareshape{stagebox}{
	%anchors
	\anchor{center}{\pgf@x=0cm \pgf@y=0cm}
	\anchor{south}{\pgf@x=0cm \pgf@y=-\boxsize}
	\anchor{east}{\pgf@x=\boxsize \pgf@y=0cm}
	\anchor{west}{\pgf@x=-\boxsize \pgf@y=0cm}
	\anchor{north}{\pgf@x=0cm \pgf@y=\boxsize}

	%draw
	\backgroundpath{
		\filldraw(-\boxsize,-\boxsize)rectangle(\boxsize,\boxsize);
    }
}
\makeatother

\makeatletter
\pgfdeclareshape{stage}{
	%anchors
	\anchor{center}{\pgf@x=0cm \pgf@y=0cm}
	\anchor{south}{\pgf@x=0cm \pgf@y=-\boxsize}
	\anchor{east}{\pgf@x=\boxsize \pgf@y=0cm}
	\anchor{west}{\pgf@x=-\boxsize \pgf@y=0cm}
	\anchor{north}{\pgf@x=0cm \pgf@y=\boxsize}

	\anchor{xout}{\pgf@x=0cm \pgf@y=-\boxsize}
	\anchor{y}{\pgf@x=\boxsize \pgf@y=0cm}
	\anchor{u}{\pgf@x=-\boxsize \pgf@y=0cm}
	\anchor{xin}{\pgf@x=0cm \pgf@y=\boxsize}

	%draw
	\backgroundpath{
		\filldraw(-\boxsize,-\boxsize)rectangle(\boxsize,\boxsize);

		%connection A
		\draw[signal][postaction = {decorate,decoration={markings,mark= at position 0.65 with {\arrow{>}}}}]
					(-\boxsize,0) -- (\boxsize,0);
		%shade at the intersection
		\filldraw[draw=none,opacity=0.7](-\shadesize,-\shadesize)rectangle(\shadesize,\shadesize);
		%connection D
		\draw[signal][postaction = {decorate,decoration={markings,mark= at position 0.35 with {\arrow{>}}}}]
					(0,\boxsize) -- (0,-\boxsize);



		%connections for B and C
		\draw[signal][postaction = {decorate,decoration={markings,mark= at position 0.6 with {\arrow{>}}}}]
					(-\rel_i_stage*\boxsize,0) -- (0,-\rel_i_stage*\boxsize);
		\draw[signal][postaction = {decorate,decoration={markings,mark= at position 0.6 with {\arrow{>}}}}]
					(0,\rel_i_stage*\boxsize) -- (\rel_i_stage*\boxsize,0);

		%Text for A,B,D,C
		\pgfsetcolor{black}
		\pgftext[right,x=-0.5em,y=0.4*\boxsize]
			{\pgfkeysvalueof{/tikz/fontmatrices} \pgfkeysvalueof{/tikz/A}}
		\pgftext[top,left,x=0.05*\boxsize,y=-0.5em]
			{\pgfkeysvalueof{/tikz/fontmatrices} \pgfkeysvalueof{/tikz/D}}
		\pgftext[left,bottom,x=0.35*\boxsize,y=0.35*\boxsize]
			{\pgfkeysvalueof{/tikz/fontmatrices} \pgfkeysvalueof{/tikz/C}}
		\pgftext[right,top,x=-0.35*\boxsize,y=-0.4*\boxsize]
			{\pgfkeysvalueof{/tikz/fontmatrices} \pgfkeysvalueof{/tikz/B}}

    }
}
\makeatother


\makeatletter
\pgfdeclareshape{stageanti}{
	%anchors
	\anchor{center}{\pgf@x=0cm \pgf@y=0cm}
	\anchor{south}{\pgf@x=0cm \pgf@y=-\boxsize}
	\anchor{east}{\pgf@x=\boxsize \pgf@y=0cm}
	\anchor{west}{\pgf@x=-\boxsize \pgf@y=0cm}
	\anchor{north}{\pgf@x=0cm \pgf@y=\boxsize}

	\anchor{xin}{\pgf@x=0cm \pgf@y=-\boxsize}
	\anchor{y}{\pgf@x=\boxsize \pgf@y=0cm}
	\anchor{u}{\pgf@x=-\boxsize \pgf@y=0cm}
	\anchor{xout}{\pgf@x=0cm \pgf@y=\boxsize}

	%draw
	\backgroundpath{
		\filldraw(-\boxsize,-\boxsize)rectangle(\boxsize,\boxsize);

		%connection A
		\draw[signal][postaction = {decorate,decoration={markings,mark= at position 0.65 with {\arrow{>}}}}]
		(-\boxsize,0) -- (\boxsize,0);
		%shade at the intersection
		\filldraw[draw=none,opacity=0.7](-\shadesize,-\shadesize)rectangle(\shadesize,\shadesize);
		%connection D
		\draw[signal][postaction = {decorate,decoration={markings,mark= at position 0.35 with {\arrow{>}}}}]
		(0,-\boxsize) -- (0,\boxsize);



		%connections for B and C
		\draw[signal][postaction = {decorate,decoration={markings,mark= at position 0.6 with {\arrow{>}}}}]
		(-\rel_i_stage*\boxsize,0) -- (0,\rel_i_stage*\boxsize);
		\draw[signal][postaction = {decorate,decoration={markings,mark= at position 0.6 with {\arrow{>}}}}]
		(0,-\rel_i_stage*\boxsize) -- (\rel_i_stage*\boxsize,0);

		%Text for A,B,D,C
		\pgfsetcolor{black}
		\pgftext[right,x=-0.4em,y=-0.4*\boxsize]
		{\pgfkeysvalueof{/tikz/fontmatrices} \pgfkeysvalueof{/tikz/A}}
		\pgftext[bottom,left,x=0.05*\boxsize,y=0.35em]
		{\pgfkeysvalueof{/tikz/fontmatrices} \pgfkeysvalueof{/tikz/D}}
		\pgftext[left,top,x=0.35*\boxsize,y=-0.35*\boxsize]
		{\pgfkeysvalueof{/tikz/fontmatrices} \pgfkeysvalueof{/tikz/C}}
		\pgftext[right,bottom,x=-0.35*\boxsize-0.1em,y=0.4*\boxsize+0.1em]
		{\pgfkeysvalueof{/tikz/fontmatrices} \pgfkeysvalueof{/tikz/B}}

	}
}
\makeatother

\begin{document}
	\maketitle
	
\section*{Introduction}

This manual explains how too draw time varying systems in Tikz.
Therefore we use custom defined shapes that are available in a seperate tex file.

First we describe the usage with the parameters.
After this we give examples.

\section*{Usage}

To load the shape we have to include these commands in the preamble.
\begin{verbatim}
	\usepackage{pgfplots}
	\usepackage{tikz}
	\usetikzlibrary{calc}
\usetikzlibrary{matrix}
\usetikzlibrary{ decorations.markings}
\usetikzlibrary {decorations.shapes}


%Standard settings for the stage4s
\tikzset{
	style_stage/.style={
		color=black!80, %line color
		draw,
		fill=white,
		line width=1pt,
	}
}

%special parametres for stages
\tikzset{fontmatrices/.initial=\small}
\tikzset{boxsize/.initial=9mm} %Note: width=height=2*boxsize
%psoition of arrow
\tikzset{posarr/.initial=0.5}

%line style for the signals
\tikzstyle{signal} = [line width=1.5pt]%[very thick]

%line style for drawing connections
\tikzset{signalflow/.style={signal,
		decoration={markings,mark=at position \pgfkeysvalueof{/tikz/posarr} with {\arrow{>}}},postaction={decorate}}}

%set the standard parameters for A,B,C,D
\tikzset{A/.initial=$A_{}$}
\tikzset{B/.initial=$B_{}$}
\tikzset{C/.initial=$C_{}$}
\tikzset{D/.initial=$D_{}$}


%Some internal definitions
\def\rel_i_stage{0.65} %distance of the connections for B and C, is a fraction of boxsize
\def\shadesize{2pt}

%some shorthanfds for convenience
\def\boxsize{\pgfkeysvalueof{/tikz/boxsize}}

\makeatletter
\pgfdeclareshape{stagebox}{
	%anchors
	\anchor{center}{\pgf@x=0cm \pgf@y=0cm}
	\anchor{south}{\pgf@x=0cm \pgf@y=-\boxsize}
	\anchor{east}{\pgf@x=\boxsize \pgf@y=0cm}
	\anchor{west}{\pgf@x=-\boxsize \pgf@y=0cm}
	\anchor{north}{\pgf@x=0cm \pgf@y=\boxsize}

	%draw
	\backgroundpath{
		\filldraw(-\boxsize,-\boxsize)rectangle(\boxsize,\boxsize);
    }
}
\makeatother

\makeatletter
\pgfdeclareshape{stage}{
	%anchors
	\anchor{center}{\pgf@x=0cm \pgf@y=0cm}
	\anchor{south}{\pgf@x=0cm \pgf@y=-\boxsize}
	\anchor{east}{\pgf@x=\boxsize \pgf@y=0cm}
	\anchor{west}{\pgf@x=-\boxsize \pgf@y=0cm}
	\anchor{north}{\pgf@x=0cm \pgf@y=\boxsize}

	\anchor{xout}{\pgf@x=0cm \pgf@y=-\boxsize}
	\anchor{y}{\pgf@x=\boxsize \pgf@y=0cm}
	\anchor{u}{\pgf@x=-\boxsize \pgf@y=0cm}
	\anchor{xin}{\pgf@x=0cm \pgf@y=\boxsize}

	%draw
	\backgroundpath{
		\filldraw(-\boxsize,-\boxsize)rectangle(\boxsize,\boxsize);

		%connection A
		\draw[signal][postaction = {decorate,decoration={markings,mark= at position 0.65 with {\arrow{>}}}}]
					(-\boxsize,0) -- (\boxsize,0);
		%shade at the intersection
		\filldraw[draw=none,opacity=0.7](-\shadesize,-\shadesize)rectangle(\shadesize,\shadesize);
		%connection D
		\draw[signal][postaction = {decorate,decoration={markings,mark= at position 0.35 with {\arrow{>}}}}]
					(0,\boxsize) -- (0,-\boxsize);



		%connections for B and C
		\draw[signal][postaction = {decorate,decoration={markings,mark= at position 0.6 with {\arrow{>}}}}]
					(-\rel_i_stage*\boxsize,0) -- (0,-\rel_i_stage*\boxsize);
		\draw[signal][postaction = {decorate,decoration={markings,mark= at position 0.6 with {\arrow{>}}}}]
					(0,\rel_i_stage*\boxsize) -- (\rel_i_stage*\boxsize,0);

		%Text for A,B,D,C
		\pgfsetcolor{black}
		\pgftext[right,x=-0.5em,y=0.4*\boxsize]
			{\pgfkeysvalueof{/tikz/fontmatrices} \pgfkeysvalueof{/tikz/A}}
		\pgftext[top,left,x=0.05*\boxsize,y=-0.5em]
			{\pgfkeysvalueof{/tikz/fontmatrices} \pgfkeysvalueof{/tikz/D}}
		\pgftext[left,bottom,x=0.35*\boxsize,y=0.35*\boxsize]
			{\pgfkeysvalueof{/tikz/fontmatrices} \pgfkeysvalueof{/tikz/C}}
		\pgftext[right,top,x=-0.35*\boxsize,y=-0.4*\boxsize]
			{\pgfkeysvalueof{/tikz/fontmatrices} \pgfkeysvalueof{/tikz/B}}

    }
}
\makeatother


\makeatletter
\pgfdeclareshape{stageanti}{
	%anchors
	\anchor{center}{\pgf@x=0cm \pgf@y=0cm}
	\anchor{south}{\pgf@x=0cm \pgf@y=-\boxsize}
	\anchor{east}{\pgf@x=\boxsize \pgf@y=0cm}
	\anchor{west}{\pgf@x=-\boxsize \pgf@y=0cm}
	\anchor{north}{\pgf@x=0cm \pgf@y=\boxsize}

	\anchor{xin}{\pgf@x=0cm \pgf@y=-\boxsize}
	\anchor{y}{\pgf@x=\boxsize \pgf@y=0cm}
	\anchor{u}{\pgf@x=-\boxsize \pgf@y=0cm}
	\anchor{xout}{\pgf@x=0cm \pgf@y=\boxsize}

	%draw
	\backgroundpath{
		\filldraw(-\boxsize,-\boxsize)rectangle(\boxsize,\boxsize);

		%connection A
		\draw[signal][postaction = {decorate,decoration={markings,mark= at position 0.65 with {\arrow{>}}}}]
		(-\boxsize,0) -- (\boxsize,0);
		%shade at the intersection
		\filldraw[draw=none,opacity=0.7](-\shadesize,-\shadesize)rectangle(\shadesize,\shadesize);
		%connection D
		\draw[signal][postaction = {decorate,decoration={markings,mark= at position 0.35 with {\arrow{>}}}}]
		(0,-\boxsize) -- (0,\boxsize);



		%connections for B and C
		\draw[signal][postaction = {decorate,decoration={markings,mark= at position 0.6 with {\arrow{>}}}}]
		(-\rel_i_stage*\boxsize,0) -- (0,\rel_i_stage*\boxsize);
		\draw[signal][postaction = {decorate,decoration={markings,mark= at position 0.6 with {\arrow{>}}}}]
		(0,-\rel_i_stage*\boxsize) -- (\rel_i_stage*\boxsize,0);

		%Text for A,B,D,C
		\pgfsetcolor{black}
		\pgftext[right,x=-0.4em,y=-0.4*\boxsize]
		{\pgfkeysvalueof{/tikz/fontmatrices} \pgfkeysvalueof{/tikz/A}}
		\pgftext[bottom,left,x=0.05*\boxsize,y=0.35em]
		{\pgfkeysvalueof{/tikz/fontmatrices} \pgfkeysvalueof{/tikz/D}}
		\pgftext[left,top,x=0.35*\boxsize,y=-0.35*\boxsize]
		{\pgfkeysvalueof{/tikz/fontmatrices} \pgfkeysvalueof{/tikz/C}}
		\pgftext[right,bottom,x=-0.35*\boxsize-0.1em,y=0.4*\boxsize+0.1em]
		{\pgfkeysvalueof{/tikz/fontmatrices} \pgfkeysvalueof{/tikz/B}}

	}
}
\makeatother

\end{verbatim}

Afterwards we can draw systems in tikz pictures.

The file provides three shapes that can be used:
\begin{itemize}
	\item \verb|stage| A stage of a causal system
	\item \verb|stageanti| A stage of a anticausal system
	\item \verb|stagebox| The border of the stage
\end{itemize}

\begin{tikzpicture}
\tikzset{
   shape example/.style={fill=yellow!30,}
}
\draw [help lines] grid (14,4);
\node(stage) [shape example, stage,A=$A$,B=$B$,C=$C$,D=$D$] at (3,2) {};
\node(stageanti) [shape example, stageanti,A=$A$,B=$B$,C=$C$,D=$D$] at (7,2) {};
\node(stagebox) [shape example, stagebox] at (11,2) {};
\node[below of=stage,yshift=-0.5cm] {\texttt{stage}};
\node[below of=stageanti,yshift=-0.5cm] {\texttt{stageanti}};
\node[below of=stagebox,yshift=-0.5cm] {\texttt{stageabox}};
\end{tikzpicture}
This graphic can be created using the following code.
Note here that the labels can be set using the keys 
\verb|[A=|$\bullet$
\verb|,B=|$\bullet$\verb|,C=|$\bullet$\verb|,D=|$\bullet$\verb|]|.
\begin{verbatim}
\begin{tikzpicture}
\tikzset{
shape example/.style={fill=yellow!30,}
}
\draw [help lines] grid (14,4);
\node(stage) [shape example, stage,A=$A$,B=$B$,C=$C$,D=$D$] at (3,2) {};
\node(stageanti) [shape example, stageanti,A=$A$,B=$B$,C=$C$,D=$D$] at (7,2) {};
\node(stagebox) [shape example, stagebox] at (11,2) {};
\node[below of=stage,yshift=-0.5cm] {\texttt{stage}};
\node[below of=stageanti,yshift=-0.5cm] {\texttt{stageanti}};
\node[below of=stagebox,yshift=-0.5cm] {\texttt{stageabox}};
\end{tikzpicture}
\end{verbatim}

If we want to connect the systems we can use the following anchors:
All shapes have the standard anchors \texttt{north, west, center, east} and \texttt{south}.
The shapes \texttt{stage} and \texttt{stageanti} have the additional anchors 
\texttt{u,y,xin} and \texttt{xout}.

\begin{tikzpicture}
\tikzset{
	shape example/.style={fill=yellow!30,}
}
\draw [help lines] grid (14,4);
\node(stage) [shape example, stage,A=$A$,B=$B$,C=$C$,D=$D$] at (3,2) {};
\foreach \anchor/\placement in {xin/above, u/left, y/right, xout/below}
\draw[shift=(stage.\anchor)] plot[mark=x] coordinates{(0,0)} node[\placement] {\scriptsize\texttt{(s.\anchor)}};

\node(stageanti) [shape example, stageanti,A=$A$,B=$B$,C=$C$,D=$D$] at (7,2) {};
\foreach \anchor/\placement in {xout/above, u/left, y/right, xin/below}
\draw[shift=(stageanti.\anchor)] plot[mark=x] coordinates{(0,0)} node[\placement] {\scriptsize\texttt{(s.\anchor)}};

\node(stagebox) [shape example, stagebox] at (11,2) {};
\foreach \anchor/\placement in {north/above, west/left, center/above, east/right, south/below}
\draw[shift=(stagebox.\anchor)] plot[mark=x] coordinates{(0,0)} node[\placement] {\scriptsize\texttt{(s.\anchor)}};
\end{tikzpicture}

To make the lines for the signal consistent, the shapes use the style \texttt{signal}.
If this style is changed a different line style can be used.

The file also provides a style \texttt{signalflow}. This style is based on \texttt{signal} and automatically adds an arrow in the middle of the line.
The position can be changed using the parameter \texttt{posarr}

\begin{tikzpicture}
\tikzstyle{signal} = [very thick]
\node(stage) [style_stage,stage,A=$A$,B=$B$,C=$C$,D=$D$] at (0,0) {};
\node(u) [left of=stage,xshift=-1.5cm] {$u$};
\node(y) [right of=stage,xshift=2cm] {$y$};
\draw[signalflow] (u) -- (stage.u);
\draw[signalflow,posarr=0.3] (stage.y) -- (y);
\end{tikzpicture}

\begin{verbatim}
\tikzstyle{signal} = [very thick]
\node(stage) [style_stage,stage,A=$A$,B=$B$,C=$C$,D=$D$] at (0,0) {};
\node(u) [left of=stage,xshift=-1.5cm] {$u$};
\draw[signalflow] (u) -- (stage.u);
\end{verbatim}

We can also customize the stage.
The standard style for a system is \verb|style_stage|. This sets the standard parameters like fillcolor, line width etc.
There are also some custom parameters:
\begin{itemize}
	\item \verb|boxsize| sets the size of the box. Parameter is half the width/height of box 
	\item \verb|fontmatrices| sets the size of a \texttt{A}, \texttt{B}, \texttt{C} and \texttt{D}
\end{itemize}



\begin{tikzpicture}
\node(stage) [style_stage, stage, boxsize=15mm,fontmatrices = \huge,  A=$A$,B=$B$,C=$C$,D=$D$] at (3,2) {};
\end{tikzpicture}
\begin{verbatim}
\begin{tikzpicture}
\node(stage) [style_stage, stage, boxsize=15mm,fontmatrices = \huge,  
    A=$A$,B=$B$,C=$C$,D=$D$] at (3,2) {};
\end{tikzpicture}
\end{verbatim}


\section*{Examples}



\section*{Appendix}
Code to illustrate anchors:
\begin{verbatim}
\begin{tikzpicture}
\tikzset{
shape example/.style={fill=yellow!30,}
}
\draw [help lines] grid (14,4);
\node(stage) [shape example, stage,A=$A$,B=$B$,C=$C$,D=$D$] at (3,2) {};
\foreach \anchor/\placement in {xin/above, u/left, y/right, xout/below}
\draw[shift=(stage.\anchor)] plot[mark=x] coordinates{(0,0)} node[\placement]
   {\scriptsize\texttt{(s.\anchor)}};

\node(stageanti) [shape example, stageanti,A=$A$,B=$B$,C=$C$,D=$D$] at (7,2) {};
\foreach \anchor/\placement in {xout/above, u/left, y/right, xin/below}
\draw[shift=(stageanti.\anchor)] plot[mark=x] coordinates{(0,0)} node[\placement]
   {\scriptsize\texttt{(s.\anchor)}};

\node(stagebox) [shape example, stagebox] at (11,2) {};
\foreach \anchor/\placement in {north/above, west/left, center/above, 
   east/right, south/below}
\draw[shift=(stagebox.\anchor)] plot[mark=x] coordinates{(0,0)} node[\placement]
   {\scriptsize\texttt{(s.\anchor)}};
\end{tikzpicture}
\end{verbatim}




\end{document}