
\documentclass[12pt]{article}
\renewcommand{\familydefault}{\sfdefault}

\newcounter{chap}
\newcounter{sec}
\newcounter{subsec}

\setcounter{chap}{1}
\setcounter{sec}{1}
\setcounter{subsec}{1}

\newcommand{\chapter}[1]{{\vspace{2em} \textbf{\Large \thechap{} #1}}\newline\vspace{2em} \stepcounter{chap} \setcounter{sec}{1} \setcounter{subsec}{1}}
\renewcommand{\section}[1]{\vspace{0.5em} \textbf{\thechap.\thesec{} #1}\newline\vspace{1em} \stepcounter{sec} \setcounter{subsec}{1}}
\renewcommand{\subsection}[1]{\textbf{\thechap.\thesec.\thesubsec{} #1} \stepcounter{subsec}}



\begin{document}
\setlength{\parindent}{0em}
\chapter{Abstract}

\begin{itemize}
	\item neural nets
	\item reduce cost of neural nets using structured matrix
	\item how to use Sequentially semisperable matrices in this context
	\item choose number of stages and tuning segmentation
	\item some sentence on results
\end{itemize}



\chapter{Introduction} 3p


\chapter{Literature Review}

\section{Neuronal Nets}


\section{Matrix structures}
eventually move it to Background


\chapter{Background/Introduction}\label{chap:background}
\begin{itemize}
	\item Sequentially semiseperable Matrices can be considered as descriptions of time varying systems.
	\item Illustrate system.
	
	\item Details on SSS systems
	
	\item Hankel Operator, Reachability, Observability Minimality
	\item Describe causal and anticausal part
	
	\item SSS as matrix factorization
\end{itemize}



\chapter{Methods} 15p

\section{Approximation of Matrices}

\subsection{Approximation algorithm}
\begin{itemize}
	\item Discuss Approximation of a general system based on cutting the singular values of the Hankel matrix.
	
	\item Discuss ordering of the sigmas, and cutting states if the sigmas are ordered.
	\item Define ordered realization as extention of the ballanced realization
	
	\item Describe how to get the sigmas for a system by transforming it.General idea: We do not have to compute the SVD of the total Hankel matrix, if we take a input-nomal system and convert it step by step to a output-normal.
	
	\item Describe the overall approximation algorithm.
\end{itemize}

\subsection{Cost of computation}
\begin{itemize}
	\item Short section on the computational cost and number of parameters.
	
	\item Define it with and without additions.
	
	\item Might also go in the Background, but I think it would be a bit to much and I did not find a detailed discussion in the literature. 
\end{itemize}

\subsection{Number of stages}
\begin{itemize}
	\item Describe continuous surrogate problem for the cost depending on the number of parameters. 
	\item Discuss educated guesses for the number of stages.
\end{itemize}

\section{Changing of Segmentation}
\begin{itemize}
	\item Describe algorithm of moving boundaries.
	\item First for causal system, then for mixed system.
	
	\item Describe splitting and combining states.
	
	\item Pseudocode for the causal systems
	
	\item Move all the proves of minimality to the appendix for brevity.
	\item Describe calculation of $\sigma$s
\end{itemize}
\section{Permutations}
if time allows add some notes on permutations here


\chapter{Experiment Setups}
4pages, probably more 2 pages
\begin{itemize}
	\item Some general notes on matrix representation.
	
	\item How define loss, images to features etc. 
	\item How measure accuracy
\end{itemize}

\chapter{Experiments}
10pages

\section{Tests of general approxiamtion}
\begin{itemize}
	\item Test different approaches to represent matrices.
	\item Have a look at the representation error in different norms for different algorithms
\end{itemize}

\section{Tests in neural Nets}
\begin{itemize}
	\item Test the different approaces in neural nets.
\end{itemize}

\chapter{Discussion}
4-5 4pages

\chapter{Conclusion}
1-2 pages

\end{document}




